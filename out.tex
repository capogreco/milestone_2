% Options for packages loaded elsewhere
\PassOptionsToPackage{unicode}{hyperref}
\PassOptionsToPackage{hyphens}{url}
\PassOptionsToPackage{dvipsnames,svgnames,x11names}{xcolor}
\documentclass[
  american,
  12pt,
]{article}
\usepackage{xcolor}
\usepackage[margin=1in]{geometry}
\usepackage{amsmath,amssymb}
\setcounter{secnumdepth}{5}
\usepackage{iftex}
\ifPDFTeX
  \usepackage[T1]{fontenc}
  \usepackage[utf8]{inputenc}
  \usepackage{textcomp} % provide euro and other symbols
\else % if luatex or xetex
  \usepackage{unicode-math} % this also loads fontspec
  \defaultfontfeatures{Scale=MatchLowercase}
  \defaultfontfeatures[\rmfamily]{Ligatures=TeX,Scale=1}
\fi
\usepackage{lmodern}
\ifPDFTeX\else
  % xetex/luatex font selection
\fi
% Use upquote if available, for straight quotes in verbatim environments
\IfFileExists{upquote.sty}{\usepackage{upquote}}{}
\IfFileExists{microtype.sty}{% use microtype if available
  \usepackage[]{microtype}
  \UseMicrotypeSet[protrusion]{basicmath} % disable protrusion for tt fonts
}{}
\usepackage{setspace}
% definitions for citeproc citations
\NewDocumentCommand\citeproctext{}{}
\NewDocumentCommand\citeproc{mm}{%
  \begingroup\def\citeproctext{#2}\cite{#1}\endgroup}
\makeatletter
 % allow citations to break across lines
 \let\@cite@ofmt\@firstofone
 % avoid brackets around text for \cite:
 \def\@biblabel#1{}
 \def\@cite#1#2{{#1\if@tempswa , #2\fi}}
\makeatother
\newlength{\cslhangindent}
\setlength{\cslhangindent}{1.5em}
\newlength{\csllabelwidth}
\setlength{\csllabelwidth}{3em}
\newenvironment{CSLReferences}[2] % #1 hanging-indent, #2 entry-spacing
 {\begin{list}{}{%
  \setlength{\itemindent}{0pt}
  \setlength{\leftmargin}{0pt}
  \setlength{\parsep}{0pt}
  % turn on hanging indent if param 1 is 1
  \ifodd #1
   \setlength{\leftmargin}{\cslhangindent}
   \setlength{\itemindent}{-1\cslhangindent}
  \fi
  % set entry spacing
  \setlength{\itemsep}{#2\baselineskip}}}
 {\end{list}}
\usepackage{calc}
\newcommand{\CSLBlock}[1]{\hfill\break\parbox[t]{\linewidth}{\strut\ignorespaces#1\strut}}
\newcommand{\CSLLeftMargin}[1]{\parbox[t]{\csllabelwidth}{\strut#1\strut}}
\newcommand{\CSLRightInline}[1]{\parbox[t]{\linewidth - \csllabelwidth}{\strut#1\strut}}
\newcommand{\CSLIndent}[1]{\hspace{\cslhangindent}#1}
\ifLuaTeX
\usepackage[bidi=basic,shorthands=off,]{babel}
\else
\usepackage[bidi=default,shorthands=off,]{babel}
\fi
\ifLuaTeX
  \usepackage{selnolig} % disable illegal ligatures
\fi
\setlength{\emergencystretch}{3em} % prevent overfull lines
\providecommand{\tightlist}{%
  \setlength{\itemsep}{0pt}\setlength{\parskip}{0pt}}
\renewcommand{\thesection}{\roman{section}}
\setcounter{section}{0}
\makeatletter
\renewcommand{\@seccntformat}[1]{\csname the#1\endcsname\quad}
\makeatother
% Indent first paragraph too; comment this line if you prefer no indent on first
\usepackage{indentfirst}
% Enforce paragraph indenting after template adjustments
\AtBeginDocument{\setlength{\parindent}{1.5em}\setlength{\parskip}{0pt}}
\usepackage{bookmark}
\IfFileExists{xurl.sty}{\usepackage{xurl}}{} % add URL line breaks if available
\urlstyle{same}
\hypersetup{
  pdftitle={Distributed Synthesis - Milestone 2},
  pdfauthor={Thomas Capogreco},
  pdflang={en-US},
  pdfkeywords={keyword1, keyword2, keyword3},
  colorlinks=true,
  linkcolor={blue},
  filecolor={Maroon},
  citecolor={blue},
  urlcolor={blue},
  pdfcreator={LaTeX via pandoc}}

\title{Distributed Synthesis - Milestone 2}
\author{Thomas Capogreco}
\date{2025-08-29}

\begin{document}
\maketitle
\begin{abstract}
This is the abstract for your milestone 2 document. Replace this with
your actual abstract.
\end{abstract}

{
\hypersetup{linkcolor=}
\setcounter{tocdepth}{3}
\tableofcontents
}
\setstretch{1.5}
\section*{Abstract}\label{abstract}
\addcontentsline{toc}{section}{Abstract}

Browser-based distributed synthesis is a novel, lithe technique for
co-located networked music performance that leverages the ubiquity,
connectivity, and computational capacity of our personal devices to
achieve multi-channel sonic works. It continues a lineage of historical
networked and participatory music performance practices that make use of
contemporaneous technology to agitate hegemonic, commodified forms of
music performance ritual. The paradigm employs the substrate of the
internet as its artistic materials, aligning the practice with the
educational project of creative coding, and entangling it in the messy
problematics of surveillance and platform capitalism. This research
employs a critical posthumanist frame to clarify an ethical position
from which creative work can be produced in this arena, with particular
focus on the themes of texture, ritual, and emergence.

\section*{Introduction}\label{introduction}
\addcontentsline{toc}{section}{Introduction}

The purpose of this document is to articulate a literature review that
situates distributed synthesis within the confluence of
material-discursive cultural practices that move through time in
parallel, but which interact with each other in a weaving, bifurcating,
and converging manner. We can understand \emph{synthesis}, for example,
to denote not merely the production of an audio signal comprised of
fluctuating voltages in a wire, but also both the productive practices
that give rise to various material forms of the synthesiser, \emph{and}
the various uses of those material forms in production of
\emph{expressive} cultural forms. Similarly, we might understand
\emph{creative coding} to denote not merely the production of specific
pixels of particular colours on a screen, but also both the cultural
practices which scaffold audiences' interactions and experiences with
creative coding \emph{and} the productive practices that give rise to
the various programming languages and computing hardware on which those
cultural practices are predicated.

As I will argue in a later chapter, the word ``expression'', when used
in these contexts, denotes the construction of a mirage - an unalienated
human subject - the performer or artist, as percieved within the context
of some type of performance or exhibition ritual.

I bring these notes forwards into the literature review so that we can
give these practices their properly ecological standing - neither the
history of synthesis, nor the history of computing (creative or
otherwise), can be fully understood as a succession of white male heroes
who made great discoveries or inventions by scouring the internal
resources of their own, encapsulated genius. Rather, I will attempt to
give a more mundane perhaps, but more realistic, and ultimately much
richer account in which these figures do not play the role of hero, but
rather of \emph{worker} - a type of touching-feeling end-node, working
intimately with the textures of materials.

This work occurs, without exception, within in the context of some
productive, self-propagating, polyphonic assemblage which, at certain
historical moments, managed to find a way to resonate to the point of
self-oscillation among an ecology of entangled cybernetic circuits
thread between and across various cultural, economic, and institutional
systems. like a mass of mycelium producing a fruiting body.
(\citeproc{ref-tsingMushroomEndWorld2017}{Tsing, 2017})

\setcounter{section}{0}

\section{Networked \& Participatory Music
Performance}\label{networked-participatory-music-performance}

\subsection{Key Concepts \& Terminology}\label{key-concepts-terminology}

\subsubsection{Networked Music Performance
(NMP)}\label{networked-music-performance-nmp}

\subsubsection{Mobile Music}\label{mobile-music}

\subsubsection{Participatory NMP}\label{participatory-nmp}

\subsubsection{Local Nework Music}\label{local-nework-music}

\subsection{Pioneering Works \&
Practitioners}\label{pioneering-works-practitioners}

\subsubsection{Pre-Smartphone Era}\label{pre-smartphone-era}

\subsubsection{The Rise of Laptop
Orchestras}\label{the-rise-of-laptop-orchestras}

\subsubsection{The Smartphone Orchestra}\label{the-smartphone-orchestra}

\subsubsection{Web-Based Participation}\label{web-based-participation}

\subsection{Frameworks \& Platforms}\label{frameworks-platforms}

\subsubsection{soundworks (IRCAM)}\label{soundworks-ircam}

\subsubsection{Collab-Hub}\label{collab-hub}

\subsubsection{PeerJS / SimplePeer}\label{peerjs-simplepeer}

\subsection{Relevance}\label{relevance}

\subsubsection{Shift in Framing}\label{shift-in-framing}

\subsubsection{Technological
Contemporaneity}\label{technological-contemporaneity}

\subsubsection{Media Archeology \& Critical Infrastructure
Studies}\label{media-archeology-critical-infrastructure-studies}

\section{Creative Coding \& Live Coding as Cultural
Practice}\label{creative-coding-live-coding-as-cultural-practice}

\subsection{Ethos of Inclusion}\label{ethos-of-inclusion}

\subsubsection{Historical Precedent}\label{historical-precedent}

\subsubsection{Foundation Texts}\label{foundation-texts}

\subsubsection{Radical Commitment to
Access}\label{radical-commitment-to-access}

\subsubsection{Pedagogy \& Community}\label{pedagogy-community}

\subsection{Live Coding as a Community of
Practice}\label{live-coding-as-a-community-of-practice}

\subsubsection{The Practice}\label{the-practice}

\subsubsection{TOPLAP \& the Manifesto}\label{toplap-the-manifesto}

\subsubsection{TidalCycles as a Vehicle}\label{tidalcycles-as-a-vehicle}

\subsection{Competing Frameworks}\label{competing-frameworks}

\subsubsection{Research-Creation}\label{research-creation}

\subsubsection{Critical Making}\label{critical-making}

\section{Practitioners of Critical
Posthumanism}\label{practitioners-of-critical-posthumanism}

\subsection{Synthesis}\label{synthesis}

\subsection{Computer Music}\label{computer-music}

\subsubsection{John Chowning of Center for Computer Research in Music \&
Acoustics
(CCRMA)}\label{john-chowning-of-center-for-computer-research-in-music-acoustics-ccrma}

\subsubsection{Miller Puckette of Institute for Research and
Coordination in Acoustics/Music
(IRCAM)}\label{miller-puckette-of-institute-for-research-and-coordination-in-acousticsmusic-ircam}

\subsubsection{James McCartney of
SuperCollider}\label{james-mccartney-of-supercollider}

\subsubsection{Alex McLean of
TidalCycles}\label{alex-mclean-of-tidalcycles}

\subsection{Eurorack}\label{eurorack}

\subsubsection{Tom Erbe of SoundHack}\label{tom-erbe-of-soundhack}

\subsubsection{Tony Rolando of
MakeNoise}\label{tony-rolando-of-makenoise}

\subsubsection{Peter Edwards of Casper
Electronics}\label{peter-edwards-of-casper-electronics}

\subsubsection{Brian Crabtree of Monome}\label{brian-crabtree-of-monome}

\subsubsection{Andrew Fitch of
NONLINEARCIRCUITS}\label{andrew-fitch-of-nonlinearcircuits}

\subsection{Creative Coding}\label{creative-coding}

\subsubsection{Rosa Menkman}\label{rosa-menkman}

\subsubsection{Dan Shiffman}\label{dan-shiffman}

\subsubsection{Lauren Lee McCarthy of
p5}\label{lauren-lee-mccarthy-of-p5}

\subsubsection{Sam Levine}\label{sam-levine}

\section{Critical Theories of Technology, \&
Sound}\label{critical-theories-of-technology-sound}

\subsection{Platform \& Surveillance
Capitalism}\label{platform-surveillance-capitalism}

\subsection{Sound, Listening, \& Power}\label{sound-listening-power}

\subsection{Media Archaeology \&
Posthumanism}\label{media-archaeology-posthumanism}

\section{Ritual, Performance, \& Media
Studies}\label{ritual-performance-media-studies}

\subsection{Performance as Ritual}\label{performance-as-ritual}

\subsection{Affect Theory \& Emergence}\label{affect-theory-emergence}

\subsection{Media Ecology \& Liveness}\label{media-ecology-liveness}

\subsubsection{Media Ecology}\label{media-ecology}

\subsubsection{Liveness in a Digital
Context}\label{liveness-in-a-digital-context}

\section*{Discussion}\label{discussion}
\addcontentsline{toc}{section}{Discussion}

Discuss the implications of your results.

\section*{Conclusion}\label{conclusion}
\addcontentsline{toc}{section}{Conclusion}

Summarize your work and future directions.

\section*{References}\label{references}
\addcontentsline{toc}{section}{References}

\protect\phantomsection\label{refs}
\begin{CSLReferences}{1}{0}
\bibitem[\citeproctext]{ref-tsingMushroomEndWorld2017}
Tsing, A. L. (2017). \emph{The {Mushroom} at the {End} of the {World}:
{On} the {Possibility} of {Life} in {Capitalist Ruins}} (Reprint
edition). Princeton University Press.

\end{CSLReferences}

\end{document}
